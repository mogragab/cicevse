\section{Introduction}
\label{section:introduction}
The electrification of transportation represents one of the most significant transformations in mobility and energy systems of the 21st century. Driven by advancements in battery technology, government incentives, and growing demand for sustainable transportation \cite{hawkins2012environmental}, electric vehicle (EV) adoption is accelerating rapidly with projections estimating approximately 230 million EVs by 2030 \cite{boudina2020review, sani2022privacy}. This unprecedented growth necessitates massive expansion of charging infrastructure to ensure user convenience and support widespread mobility electrification. \\

%## Infrastructure Complexity and Connectivity
Modern Electric Vehicle Supply Equipment (EVSE) has evolved into sophisticated cyber-physical systems that integrate complex hardware and software components to manage charging operations safely and efficiently \cite{buedi2024enhancing}. To enhance user experience and operational efficiency, contemporary charging infrastructure incorporates advanced communication capabilities through protocols such as the Open Charge Point Protocol (OCPP), which facilitates remote communication between EVSE and central Charging Station Management Systems (CSMS) \cite{makhmudov2025online}. Additionally, standards like ISO 15118 enable Vehicle-to-Grid (V2G) communication for advanced functionalities including Plug \& Charge authentication \cite{naeem2025optimizing, buedi2024enhancing}. This integration creates a complex ecosystem where EVs are embedded within smart grids and Vehicle-to-Everything (V2X) networks \cite{makhmudov2025online}. The convergence of power electronics, including battery management systems and motor control units, with Internet of Things (IoT) connectivity establishes multiple potential entry points for malicious actors \cite{almadhor2025transfer}. \\

%## Cybersecurity Vulnerabilities and Threat Landscape
The enhanced connectivity and functionality that improve operational convenience simultaneously introduce significant cybersecurity vulnerabilities \cite{hamdare2023cybersecurity}. EV charging stations function as critical nodes at the intersection of transportation networks, electrical grids, and digital communication systems, creating a complex attack surface that traditional cybersecurity approaches struggle to address effectively. Features designed to improve user experience, such as remote monitoring capabilities and diverse authentication methods including RFID, NFC, and QR codes, create potential attack vectors \cite{antoun2020detailed}. \\

The threat landscape encompasses various attack vectors targeting both in-vehicle networks and charging infrastructure. Controller Area Network (CAN) bus systems and charging infrastructure are vulnerable to spoofing attacks, denial-of-service incidents, malware infiltration, and firmware tampering \cite{rai2025securing, almadhor2025transfer}. Open EVSE protocols and networked battery management systems present additional vulnerabilities to remote compromise \cite{lipu2023artificial}. Recent security incidents demonstrate the severity of these vulnerabilities. The Brokenwire attack (CVE-2022-0878) successfully disrupted seven vehicles and eighteen charging stations using less than one watt of power from distances up to 47 meters \cite{kohler2022brokenwire}. Security researchers have identified six zero-day vulnerabilities across sixteen live charging management systems that could enable remote station shutdown and energy theft. These incidents underscore the urgent need for advanced, real-time detection systems capable of identifying and mitigating cyber threats before they compromise critical infrastructure \cite{johnson2022review}. Malicious actors can exploit these vulnerabilities to launch sophisticated attacks including Distributed Denial of Service (DDoS), data theft, and manipulation of charging processes, posing significant risks to user safety, grid stability, and privacy \cite{kumar2025machine, alcaraz2017ocpp}. \\

%## Machine Learning Approaches for Cybersecurity
Machine learning offers promising solutions for detecting and mitigating cybersecurity threats in real-time within EV ecosystems. Unlike static rule-based methods, machine learning models can learn patterns of both normal and malicious behavior, enabling effective anomaly-based intrusion detection and continuous system monitoring \cite{sani2022privacy, mohamed2023artificial}. Researchers have successfully applied machine learning techniques to various aspects of EV cybersecurity, including CAN bus intrusion detection \cite{bari2023intrusion, rai2025securing}, EV charging network intrusion detection systems \cite{almadhor2025transfer, makhmudov2025online}, and battery management system anomaly detection. \\

However, traditional intrusion detection systems have limited applicability in EV contexts due to privacy concerns and the heterogeneous nature of charging environments. Federated learning presents a promising solution by enabling collaborative model training through aggregation of local model updates without requiring raw data sharing, thus preserving privacy while maintaining security effectiveness. \\

%## Dataset Availability and Research Opportunities
A critical challenge in developing effective machine learning solutions for EV charging infrastructure security has been the lack of comprehensive, realistic datasets. This gap has been addressed by the recent publication of the CICEVSE2024 dataset \cite{buedi2024enhancing}, which provides multi-dimensional cybersecurity data captured from real EVSE testbeds under both benign and attack conditions. The dataset encompasses power consumption data, network traffic patterns, and host-level activities including Hardware Performance Counters and kernel events. It includes diverse modern attack scenarios such as reconnaissance scans, DoS floods, cryptojacking, and backdoor attacks, enabling researchers to develop, train, and validate sophisticated security solutions with unprecedented fidelity. \\

%## Research Contribution and Approach
This paper addresses the cybersecurity challenges in EV charging infrastructure through a novel application of Temporal Convolutional Networks (TCN) for cyber attack detection. TCNs offer significant advantages over traditional Recurrent Neural Networks, including parallel processing capabilities, stable gradient flow, and exponential receptive field growth through dilated convolutions. Our approach leverages the CICEVSE2024 dataset to develop an adaptive federated deep-learning intrusion detection system specifically designed for EV charging infrastructure. \\

The key contributions of this work include the design of a decentralized deep learning model, implementation of concept drift detection mechanisms, and comprehensive evaluation using real-world datasets. Our system enables collaborative threat detection while preserving privacy through federated learning principles, addressing the unique requirements of heterogeneous EV charging environments.

\begin{comment}
The rapid proliferation of \gls{ev} adoption and associated charging infrastructure represents one of the most significant transformations in transportation and energy systems of the 21st century. The electrification of transportation and adoption of \glspl{ev} is accelerating rapidly; projections estimate ~230 million \glspl{ev} by 2030 \cite{boudina2020review, sani2022privacy}. a shift motivated by advancements in battery technology, government incentives, and a growing demand for sustainable transportation \cite{hawkins2012environmental}. This proliferation of EVs necessitates a massive expansion of the corresponding charging infrastructure to ensure user convenience and support mobility. The cybersecurity implications of this infrastructure expansion demand immediate and comprehensive attention. \gls{ev} charging stations function as critical nodes in the intersection between transportation networks, electrical grids, and digital communication systems, creating a complex attack surface that traditional cybersecurity approaches struggle to address effectively. \\

To support this growth, \glspl{ev} are integrated into smart grids and \gls{v2x} networks, using standards like \gls{ocpp} for charging station communication \cite{makhmudov2025online}. However, this connectivity introduces significant cybersecurity risks. \gls{ev} in-vehicle networks \gls{can} and charging infrastructure can be attacked by spoofing, denial-of-service, malware, and firmware tampering \cite{rai2025securing, almadhor2025transfer}. The convergence of power electronics (battery, motor control) with IoT connectivity gives attackers new entry points \cite{almadhor2025transfer}. For example, open EVSE protocols and networked \gls{bms} are vulnerable to remote compromise \cite{lipu2023artificial}. \\

The modern Electric Vehicle Supply Equipment (EVSE) is a complex cyber-physical system, integrating sophisticated hardware and software to manage charging operations safely and efficiently \cite{buedi2024enhancing}. To enhance user experience, the charging infrastructure incorporates advanced communication features. Protocols like the Open Charge Point Protocol (OCPP) facilitate remote communication between the EVSE and a central Charging Station Management System (CSMS), while standards such as ISO 15118 enable Vehicle-to-Grid (V2G) communication for functionalities like Plug\&Charge \cite{naeem2025optimizing, buedi2024enhancing}.

However, this increased connectivity and functionality introduce significant cybersecurity vulnerabilities \cite{hamdare2023cybersecurity}. The very features designed to improve convenience, such as remote monitoring and diverse authentication methods (e.g., RFID, NFC, QR codes), create potential attack surfaces \cite{antoun2020detailed}. The Brokenwire attack (CVE-2022-0878) successfully disrupted 7 vehicles and 18 charging stations using less than 1W of power from distances up to 47 meters, while security researchers have identified six zero-day vulnerabilities across 16 live charging management systems that could enable remote station shutdown and energy theft \cite{kohler2022brokenwire}. 

These incidents underscore the urgent need for advanced, real-time detection systems capable of identifying and mitigating cyber threats before they can compromise critical infrastructure \cite{johnson2022review}. Malicious actors can exploit these vulnerabilities to launch a variety of attacks, including Denial of Service (DoS), data theft, and even manipulating the charging process, posing risks to user safety, grid stability, and privacy \cite{kumar2025machine, alcaraz2017ocpp}.
\newpage

The rapid adoption of \glspl{ev} and the proliferation of \gls{ev} charging infrastructure introduce critical cybersecurity challenges. Charging stations and the \gls{ev} ecosystem are increasingly connected, making them vulnerable to network-based attacks such as \gls{ddos} and malware intrusions. \gls{ids} have limited applicability in \gls{ev} contexts due to privacy concerns and heterogeneous environments. \gls{fl} offers a promising solution: it allows collaborative model training by aggregating local model updates without sharing raw data. In this study, we develop an adaptive federated deep-learning \gls{ids} for \gls{ev} and \gls{ev} charging infrastructure. Our system is evaluated on publicly available \gls{ev}-focused datasets. Key contributions include design of a decentralized deep model, use of concept drift detection, and real-world evaluation. \\

\gls{ml} offers a promising means to detect and mitigate such threats in real time. Unlike static rule-based methods, \gls{ml} models (supervised, unsupervised, deep, online, federated, etc.) can learn patterns of both normal and malicious behavior, enabling anomaly-based intrusion detection and system monitoring \cite{sani2022privacy, mohamed2023artificial}. In \gls{ev} contexts, researchers have applied \gls{ml} to \gls{can} bus intrusion detection \cite{bari2023intrusion, rai2025securing}, \gls{ev} charging network IDS \cite{almadhor2025transfer, makhmudov2025online}, \gls{bms} anomaly detection,  \gls{bms} anomaly detection, and more. This report organizes the literature and case studies by security domain, details methodologies, and highlights practical implementations associated with 2024 research (including CICEVSE2024 datasets and workshops). We also address privacy in telematics, adversarial attacks on \gls{ml} models, and outline open challenges and future directions. \\

This critical gap is addressed by the recent publication of the CICEVSE2024 dataset \cite{buedi2024enhancing}. It is a comprehensive, multi-dimensional dataset that captures power consumption data, network traffic, and host-level activities (Hardware Performance Counters and kernel events) from a real EVSE testbed under both benign and attack conditions. The dataset includes a wide range of modern attack scenarios, such as reconnaissance scans, DoS floods, cryptojacking, and backdoor attacks. The availability of such a rich dataset enables researchers to develop, train, and validate sophisticated ML/DL-based security solutions with unprecedented fidelity. This work aims to leverage the CICEVSE2024 dataset to analyze and propose advanced ML-driven frameworks for enhancing the security and resilience of the EV charging ecosystem.

This paper addresses these challenges through a novel application of \gls{tcn} for cyber attack detection in \gls{ev} charging stations. \glspl{tcn} offer significant advantages over traditional \glspl{rnn} including parallel processing capabilities, stable gradient flow, and exponential receptive field growth through dilated convolutions. Our approach leverages the recently released CICEVSE2024 dataset, which provides multi-dimensional cybersecurity data including network traffic patterns, hardware performance counters, and power consumption measurements from real \gls{ev} charging stations.

\begin{comment}

\subsection{Research Contributions}
This research makes several key contributions to the cybersecurity and smart grid security domains:
\begin{itemize}
	\item \textbf{Novel TCN Architecture:} Development of a specialized TCN architecture incorporating attention mechanisms and residual connections optimized for EV charging station attack detection.
	\item \textbf{Comprehensive Experimental Analysis:} Systematic comparison with LSTM, GRU, CNN, and hybrid approaches using standardized evaluation protocols.
	\item \textbf{Real-time Deployment Framework:} Practical implementation strategy for resource-constrained edge computing environments.
	\item \textbf{Multi-dimensional Feature Engineering:} Advanced preprocessing techniques for heterogeneous EV charging station data.
	\item \textbf{Imbalanced Dataset Handling:} Application of ADASYN and other advanced techniques for addressing class imbalance in cybersecurity datasets.
\end{itemize}

The remainder of this paper is organized as follows: Section 2 reviews related work in network intrusion detection and EV cybersecurity; Section 3 presents our TCN-based methodology; Section 4 details experimental setup and evaluation protocols; Section 5 presents comprehensive results and analysis; Section 6 discusses implications and deployment considerations; and Section 7 concludes with future research directions.
\end{comment}