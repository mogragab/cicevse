% literature review 
\section{Related work}
\label{section:related}
The cybersecurity landscape for electric vehicles and charging infrastructure has evolved significantly, encompassing multiple domains from in-vehicle networks to charging station management systems. This section examines the current state of research across key areas relevant to EV cybersecurity, highlighting the progression from traditional security approaches to advanced machine learning solutions. \\

%## Vehicular Network Security and Intrusion Detection
Modern electric vehicles incorporate numerous Electronic Control Units (ECUs) communicating through various bus systems, including Controller Area Network (CAN) and Ethernet protocols, which lack built-in authentication mechanisms and remain vulnerable to sophisticated attacks. Early research in vehicular network security focused on developing intrusion detection systems for CAN bus communications. Kang et al. pioneered the application of Deep Neural Networks for distinguishing between normal and abnormal CAN messages \cite{kang2016intrusion}. \\

Recognizing the sequential nature of vehicular communications, subsequent research advanced toward temporal modeling approaches. Taylor et al. and Ashraf et al. employed Recurrent Neural Networks and Long Short-Term Memory networks to detect message injection attacks by predicting subsequent messages in communication sequences \cite{taylor2016anomaly, ashraf2020novel}. More recent developments have adapted Convolutional Neural Networks for this domain by transforming tabular or sequential CAN bus data into image formats, utilizing techniques such as stacked one-hot encoded CAN identifiers and recurrence plots \cite{seo2018gids, desta2022rec}. \\

However, these systems face significant limitations, particularly in their vulnerability to adversarial attacks. Aloraini et al. demonstrated that substitute-model attacks could dramatically reduce an in-vehicle intrusion detection system's F1-score from approximately 95\% to 38\% through strategic perturbation of CAN messages \cite{aloraini2024adversarial}. This vulnerability underscores the critical need for robust and explainable machine learning methodologies in vehicular cybersecurity applications. \\

%## EV Charging Infrastructure Security
The security of EV charging infrastructure represents a distinct challenge within the broader EV ecosystem. Comprehensive security assessments have identified vulnerabilities spanning communication protocols, upstream services, and physical hardware components \cite{antoun2020detailed, pourmirza2021electric}. The Open Charge Point Protocol (OCPP), serving as the de facto standard for Electric Vehicle Supply Equipment to Charging Station Management System communication, has emerged as a significant vulnerability point.

OCPP's reliance on unencrypted WebSocket connections in versions such as 1.6 creates susceptibility to Man-in-the-Middle attacks, remote code execution, and Denial of Service threats \cite{elmo2023disrupting, alcaraz2017ocpp}. Extended analyses of newer OCPP versions have revealed persistent security challenges, highlighting the necessity for specialized defense mechanisms tailored to the unique operational technology environment of charging infrastructure \cite{alcaraz2023ocpp}.

\newpage
%## Machine Learning Applications in EVSE Security
The application of machine learning to EV charging station security has accelerated with the availability of specialized datasets. The development of the CICEVSE2024 dataset represents a significant milestone, providing multi-dimensional cybersecurity data captured from physical EVSE testbeds under both benign and attack conditions \cite{buedi2024enhancing}. This dataset encompasses synchronized network traffic, power consumption data, and fine-grained host-level events, enabling comprehensive security analysis across multiple system dimensions. \\

Kumar et al. leveraged this dataset to develop dual detection models: a Host Anomaly Detection Model utilizing Hardware Performance Counters and kernel events, and a Power Anomaly Detection Model analyzing power consumption patterns \cite{kumar2025machine}. Their evaluation of algorithms including Isolation Forest, Autoencoders, XGBoost, and Transformers demonstrated the effectiveness of Transformer architectures in achieving high accuracy across different data modalities. \\

Building upon these foundations, Naeem et al. introduced sophisticated deep transfer learning frameworks that transform network traffic data into image representations for analysis by pre-trained Convolutional Neural Network architectures including Xception, VGG19, and Inception \cite{naeem2025optimizing}. Their approach incorporates Genetic Algorithm-based hyperparameter optimization and ensemble methods, achieving near-perfect accuracy in multi-class attack classification. Almadhor et al. further demonstrated the potential of transfer learning combined with deep neural networks, achieving approximately 97\% accuracy on CICEVSE2024 attack detection tasks \cite{almadhor2025transfer}. \\

The dynamic nature of cybersecurity threats has driven research toward adaptive detection systems. Makhmudov et al. proposed online intrusion detection systems for EV chargers utilizing Adaptive Random Forest algorithms with concept drift detection, emphasizing the importance of adapting to evolving attack patterns \cite{makhmudov2025online}. \\

%## Battery Management System Security
Battery Management System security represents a critical component of EV cybersecurity, as BMS compromise can lead to safety failures including overheating and fire hazards. Machine learning-based anomaly detection has been applied to BMS telemetry data encompassing voltage, current, and temperature measurements. Park et al. conducted comparative analysis of unsupervised methods on real BMS datasets, finding that Isolation Forest outperformed Local Outlier Factor for detecting current and temperature anomalies, achieving 99.43\% accuracy on current data analysis \cite{lipu2023artificial}. Beyond anomaly detection, researchers have proposed formal security models and hardware modifications, including encryption implementation in battery sensors, to protect BMS from spoofed sensor data and firmware tampering attacks \cite{lipu2023artificial}. \\

%## Vehicle-to-Everything Communication Security
Vehicle-to-Everything communications extend EV networks into roadside infrastructure and inter-vehicular communications, creating additional security challenges related to message authenticity and privacy preservation. Recent research has explored the integration of machine learning with decentralized architectures for V2X security enhancement. Zhou et al. proposed federated learning approaches among vehicles to collaboratively train V2X models without requiring raw data sharing, addressing both privacy concerns and data heterogeneity challenges \cite{sani2022privacy}. In such frameworks, individual vehicles or Roadside Units train local models for anomaly detection or misbehavior detection and share model updates for collaborative aggregation. Raja et al. further enhanced these approaches by integrating differential privacy noise into federated V2X intrusion detection systems to guard against data poisoning while maintaining detection accuracy \cite{javed2014adversarial}. \\

%## Privacy-Preserving Machine Learning in EV Systems
The application of machine learning to EV data for usage forecasting, route planning, and security monitoring raises significant privacy concerns. Research has emphasized privacy-preserving machine learning techniques specifically tailored for EV applications \cite{mousaei2024advancing}. Federated learning naturally addresses these concerns by limiting raw data sharing and has been successfully applied to EV charging behavior prediction and route planning scenarios. Differential privacy techniques, which add statistical noise to datasets during model training, provide additional privacy protection for telematics data while maintaining model utility \cite{mazhar2023electric}. Recent surveys indicate that approximately 20\% of EV machine learning research focuses on privacy and authentication issues, with emerging trends toward combining blockchain technologies with machine learning for secure model sharing \cite{khan2023blockchain}. \\

%## Research Gaps and Opportunities
Despite significant progress in EV cybersecurity research, several limitations persist in current approaches. Many existing intrusion detection systems are constrained to binary classification scenarios, struggling with the more complex challenge of multi-class attack type identification \cite{naeem2025optimizing}. Additionally, the heterogeneous nature of EV charging environments and privacy concerns limit the applicability of traditional centralized intrusion detection systems. Federated learning emerges as a promising solution that enables collaborative model training through aggregation of local model updates without requiring raw data sharing, thus preserving privacy while maintaining security effectiveness. However, the application of federated learning to EV cybersecurity remains in its nascent stages, presenting significant opportunities for advancement in distributed security frameworks specifically designed for the unique requirements of EV charging infrastructure.

\begin{comment}

Previous studies on \acrfull{ev} security have shown that charging stations are highly susceptible to cyber threats. Several approaches have used machine learning for anomaly detection in these environments.
\acrfull{fl} has gained prominence for enabling privacy-preserving distributed training in healthcare, finance, and \gls{iot} systems, but its application to \gls{ev} cybersecurity remains nascent. Intrusion Detection in Vehicular Networks (CAN/VANET): Modern \glspl{ev} comprise many ECUs communicating over bus systems (CAN, Ethernet) without built-in authentication, leaving them open to attacks. ML-based IDS have been developed to protect these networks.

These works demonstrate that both shallow \gls{ml} and deep models can effectively classify vehicular network anomalies. However, attackers can employ adversarial techniques to evade ML-based \gls{ids}. Aloraini et al. show that a substitute-model attack can dramatically drop an in-vehicle IDS’s F1-score (from ~95\% to 38\%) by perturbing CAN messages, underscoring the need for robust and explainable ML methods \cite{aloraini2024adversarial}. IDS for \gls{ev} Charging Infrastructure (EVSE): Charging networks form a critical part of the \gls{ev} ecosystem, vulnerable to network/host attacks (e.g. DoS, spoofing). A notable development is the CICEVSE2024 dataset, which includes network traffic, host logs, and hardware metrics from two EVSE types under benign and attack scenarios
\cite{almadhor2025transfer, buedi2024enhancing}. \\

The security of the EV charging ecosystem has been a growing concern. Studies have provided detailed security assessments, identifying vulnerabilities in communication protocols, upstream services, and physical hardware \cite{antoun2020detailed, pourmirza2021electric}. The Open Charge Point Protocol (OCPP), a de facto standard for EVSE-to-CSMS communication, has been identified as a significant point of weakness. Its reliance on unencrypted WebSocket connections in versions like 1.6 makes it susceptible to Man-in-the-Middle (MITM) attacks, remote code execution, and DoS threats \cite{elmo2023disrupting, alcaraz2017ocpp}. More recent analyses have extended this to newer versions, highlighting persistent threats and proposing countermeasures \cite{alcaraz2023ocpp}. These findings underscore the necessity of moving beyond traditional IT security and developing specialized defense mechanisms for the unique operational technology (OT) environment of EVSEs.

Intrusion Detection Systems (IDS) are a primary line of defense against cyberattacks. In the context of vehicular networks, early research focused on the Controller Area Network (CAN) bus. Initial approaches used Deep Neural Networks (DNNs) to distinguish between normal and abnormal CAN messages \cite{kang2016intrusion}. Recognizing the sequential nature of CAN communications, subsequent research employed Recurrent Neural Networks (RNNs) and Long Short-Term Memory (LSTM) networks to detect message injection attacks by predicting subsequent messages in a sequence \cite{taylor2016anomaly, ashraf2020novel}.

More recently, Convolutional Neural Networks (CNNs) have been adapted for this task. Since CNNs are designed for grid-like data, this approach requires transforming the tabular or sequential CAN bus data into an image format. For instance, researchers have used stacked one-hot encoded CAN IDs or recurrence plots to create image representations for CNN-based IDS \cite{seo2018gids, desta2022rec}. While effective, many of these systems were limited to binary classification (i.e., normal vs. attack) and struggled with the more complex challenge of multi-class classification to identify specific attack types \cite{naeem2025optimizing}.

The development of data-driven security solutions is contingent on the availability of high-quality datasets. For a long time, the EV research community lacked datasets specifically tailored for cybersecurity. Publicly available collections such as ACN-Data \cite{lee2019acn}, Pecan Street Dataport, and ELaadNL were generated to study the impact of EV charging on the power grid and focused on features like power consumption and charging session duration \cite{buedi2024enhancing}. While valuable for load forecasting, they are unsuitable for training intrusion detection models as they do not contain instances of cyberattacks.

A step toward cybersecurity-focused data was the CICEV2023 dataset, which simulated DDoS attacks targeting EV authentication in a controlled environment \cite{kim2023ddos}. However, the CICEVSE2024 dataset represents a major advancement \cite{buedi2024enhancing}. Created using a physical EVSE testbed, it provides a multi-dimensional view of the system's state by capturing synchronized network traffic, power consumption, and fine-grained host-level events. Its inclusion of diverse, realistic attack scenarios makes it an invaluable resource for developing and benchmarking the next generation of EVSE security solutions.

Using this, researchers have applied \gls{ml} to \gls{evcs} security. For instance, transfer learning combined with deep neural networks achieved ~97\% accuracy on CICEVSE2024 attack detection \cite{almadhor2025transfer}. Naor et al. proposed an online \gls{ids} for \gls{ev} chargers using Adaptive Random Forest with drift detection, highlighting the need to adapt to evolving attack patterns \cite{makhmudov2025online, makhmudov2025online}. Deep learning and hybrid models (e.g. CNN+GRU, GAN+LSTM) have also been explored for predicting the “remaining useful life” of an attack or for multiclass attack classification on \gls{evse} data \cite{buedi2024enhancing}. \\

These studies demonstrate practical ML implementations in EVSE networks, leveraging both legacy techniques (RF, SVM) and advanced DL architectures. \gls{bms} Security: The \gls{bms} regulates cell charging, so its compromise can lead to safety failures (overheating, fires). ML-based anomaly detection is thus applied to \gls{bms} telemetry (voltage, current, temperature). Park et al. compare unsupervised methods on a real \gls{bms} dataset, finding that Isolation Forest outperforms Local Outlier Factor for detecting current/temperature anomalies (e.g. 99.43\% accuracy on current data). These results suggest robust detection of battery anomalies is feasible with ML. Beyond anomaly detection, formal security models and hardware changes (encryption in battery sensors) are also suggested to protect \gls{bms} from spoofed sensor data or firmware tampering
\cite{lipu2023artificial}. \\

With the advent of the CICEVSE2024 dataset, researchers have begun to apply a range of machine learning models to enhance EVSE security. Kumar et al. \cite{kumar2025machine} proposed two distinct models: a Host Anomaly Detection Model (HADM) using HPC and kernel events, and a Power Anomaly Detection Model (PADM) using power consumption data. Their work evaluated a suite of algorithms, including Isolation Forest, Autoencoders, XGBoost, and Transformers, demonstrating that models like the Transformer can achieve high accuracy even with different data modalities.

Building on this, Naeem et al. \cite{naeem2025optimizing} introduced a more sophisticated framework using deep transfer learning. Their approach transforms network traffic data from CICEVSE2024 into images and applies pre-trained CNN architectures (Xception, VGG19, Inception). Crucially, they employ a Genetic Algorithm (GA) for hyperparameter optimization and use ensemble methods to combine model outputs, achieving near-perfect accuracy in multi-class attack classification. This line of research highlights a clear trend towards leveraging complex, optimized deep learning architectures to build highly effective and resilient security systems for the critical infrastructure of electric vehicles.

Overall, ML provides an additional layer of defense for \gls{ev} power systems by catching abnormal states that may signal attacks or faults. \gls{v2x} communications extend \gls{ev} networks into roadside infrastructure and other vehicles. Ensuring message authenticity and privacy is crucial. Recent work explores combining \gls{ml} with decentralized architectures for \gls{v2x} security. Zhou et al. propose using \gls{fl} among vehicles to collaboratively train \gls{v2x} models without sharing raw data, addressing privacy and data heterogeneity. In such a setup, individual \glspl{ev} (or Roadside Units) train local models (e.g. for anomaly detection or misbehavior detection) and share model updates for aggregation. Similarly, distributed \gls{ids} frameworks using \gls{admm} and FedAvg have been applied to \gls{v2x} networks to create collaborative \gls{ids} that protect both detection performance and data privacy \cite{sani2022privacy}. For example, Raja et al. integrate DP noise into a federated \gls{v2x} \gls{ids} to guard against data poisoning while maintaining accuracy~\cite{javed2014adversarial}.  \\

\acrlong{ml} on this data (for usage forecasting, route planning, etc.) raises privacy issues. Several works emphasize privacy-preserving \gls{ml} techniques for \glspl{ev} \cite{mousaei2024advancing}. \acrlong{fl} naturally limits raw data sharing and has been applied to \gls{ev} charging behavior prediction and route planning scenarios. Differential privacy (adding noise to datasets) can also be applied to telematics data when training \gls{ml} models, mitigating privacy leakage \cite{mazhar2023electric}. For example, in \gls{ev} intrusion detection, a participant’s data can be obfuscated while still contributing to a global model. A recent survey notes that about 20\% of \gls{ev} ML research focuses on privacy/authentication issues and highlights a trend toward combining blockchain with ML for secure model sharing \cite{khan2023blockchain}.
\end{comment}
